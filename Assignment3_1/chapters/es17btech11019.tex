\begin{abstract}
This document contains a problem based on properties of triangle.
\end{abstract}

Download the python codes from 
%
\begin{lstlisting}
https://github.com/sahilsin/MatrixTheory/tree/master/Assignment3_1/codes
\end{lstlisting}
%
and latex-tikz codes from 
%
\begin{lstlisting}
https://github.com/sahilsin/MatrixTheory/tree/master/Assignment3_1/figs
\end{lstlisting}
%
\section{PROBLEM}
In triangle PQR , $PR>PQ$ and PS bisects $\angle QPR$.Prove that $\angle PSR > \angle PSQ$. 

\section{SOLUTION}

\textbf{Given:} $PR>PQ$ and $\angle QPS = \angle RPS $\\
\textbf{To Prove:} $\angle PSR = \angle PSQ $\\
\textbf{Proof:}\\

As PS bisects $\angle QPR$
\begin{align}
    \angle QPS = \angle RPS \label{eq:1}
\end{align}

Using property angle opposite to larger side is always larger.

\begin{align}
    \angle PQR > \angle PRQ \label{eq:2}
\end{align}

Using property of sum of exterior angle is equal to sum of opposite interior angles.
\begin{align}
    \angle PSR = \angle PQR + \angle QPS
\\
    \angle PSQ = \angle RPS + \angle PRQ
\end{align}
 
 Adding \ref{eq:1} and \ref{eq:2} and using above properties we get :
 \begin{align}
     \angle PQR + \angle QPS > \angle PRQ + \angle RPS
\\
    \angle PSR > \angle PSQ
 \end{align}

\begin{figure}[h!]
\begin{center}
	\resizebox{\columnwidth/1}{!}{\input{figs/traingle}}
\end{center}
\caption{Triangle}
\label{fig:Triangle}
\end{figure}

