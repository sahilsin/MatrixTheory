\section{Problem}
\begin{enumerate}
    \item Consider a Markov chain with transition probability matrix P given by
    \begin{align}
        \myvec{\cfrac{1}{2}&\cfrac{1}{2}&0\\0&\cfrac{1}{2}&\cfrac{1}{2}\\\cfrac{1}{3}&\cfrac{1}{3}&\cfrac{1}{3} }
    \end{align}
    For any two states i and j.Let $P_{ij}^{(n)}$ denote the n step transition probability of going form i to j. Identify Correct statements.
    \begin{align}
        i)\lim_{n\to\infty} P_{11}^{(n)}=\cfrac{2}{9}
    \\
        ii)\lim_{n\to\infty} P_{21}^{(n)}=0
    \\
        iii)\lim_{n\to\infty} P_{32}^{(n)}=\cfrac{1}{3}
    \\
        iv)\lim_{n\to\infty} P_{13}^{(n)}=\cfrac{1}{3}
    \end{align}
\section{solution}
Using Theorem : If a finite Markov Chain is irreducible and aperiodic then there is a unique stationary distribution
\begin{align}
    \lim_{k\to\infty} P^{(k)}=Ik
\end{align}
where I represents a column vector with each entry as 1. For 3 states we have :
\begin{align}
    \lim_{n\to\infty}P^{(n)}=\myvec{1\\1\\1}\myvec{\pi_{1}&\pi_{2}&\pi_{3}}
\\
    =\myvec{\pi_{1}&\pi_{2}&\pi_{3}\\\pi_{1}&\pi_{2}&\pi_{3}\\\pi_{1}&\pi_{2}&\pi_{3}}
\\
    \implies \lim_{n\to\infty}\myvec{P_{11}^{(n)}&P_{12}^{(n)}&P_{13}^{(n)}\\P_{21}^{(n)}&P_{22}^{(n)}&P_{23}^{(n)}\\P_{31}^{(n)}&P_{32}^{(n)}&P_{33}^{(n)}}
\\
    =\myvec{\pi_{1}&\pi_{2}&\pi_{3}\\\pi_{1}&\pi_{2}&\pi_{3}\\\pi_{1}&\pi_{2}&\pi_{3}}
\end{align}

A stationary distribution $\pi$ is a row vector whose entries are non-negative and sums up to 1, is unchanged by the operation of transition matrix p on it and so it is defined by $\pi P = \pi$.
Let
\begin{multline}
    \pi=\myvec{\pi_{1}&\pi_{2}&\pi_{3}}
\\
    \implies \myvec{\pi_{1}&\pi_{2}&\pi_{3}}\myvec{\cfrac{1}{2}&\cfrac{1}{2}&0\\0&\cfrac{1}{2}&\cfrac{1}{2}\\\cfrac{1}{3}&\cfrac{1}{3}&\cfrac{1}{3}} = \myvec{\pi_{1}&\pi_{2}&\pi_{3}}
\\
    \implies \myvec{ \cfrac{\pi_{1}}{2}+\cfrac{\pi_{3}}{3}& \cfrac{\pi_{1}}{2}+\cfrac{\pi_{2}}{2}+\cfrac{\pi_{3}}{3}&\cfrac{\pi_{2}}{2}+\cfrac{\pi_{3}}{3}}=\myvec{\pi_{1}&\pi_{2}&\pi_{3}}
\\
    \implies \myvec{ \cfrac{\pi_{1}}{2}+\cfrac{\pi_{3}}{3}& \cfrac{\pi_{1}}{2}+\cfrac{\pi_{2}}{2}+\cfrac{\pi_{3}}{3}&\cfrac{\pi_{2}}{2}+\cfrac{\pi_{3}}{3}}-\myvec{\pi_{1}&\pi_{2}&\pi_{3}}
\\
    =0
\\
    \implies\myvec{\cfrac{\pi_{1}}{2}+\cfrac{\pi_{3}}{3}-\pi_{1}& \cfrac{\pi_{1}}{2}+\cfrac{\pi_{2}}{2}+\cfrac{\pi_{3}}{3}-\pi_{2}&\cfrac{\pi_{2}}{2}+\cfrac{\pi_{3}}{3}-\pi_{3}}
\\
    =0
\\
    \implies \myvec{-3\pi_{1}+2\pi_{3}&3\pi_{1}-3\pi_{2}+2\pi_{3}&3\pi_{2}-4\pi_{3}}=0
\end{multline}
\begin{align}
    -3\pi_{1}+2\pi_{3}=0 
\\
     3\pi_{1}-3\pi_{2}+2\pi_{3}=0 
\\
     3\pi_{2}-4\pi_{3} =0 \And \pi_{1}+\pi_{2}+\pi_{3}=1 
\end{align}

These equations can be transformed into the matrix :
\begin{multline}
    \myvec{-3&0&2\\3&-3&2\\0&3&-4\\1&1&1}\Vec{\pi}=\myvec{0\\0\\0\\1}
\\
    \implies\xleftrightarrow{R_{2}=R_{1}+R_{2}}\myvec{-3&0&2\\0&-3&4\\0&3&-4\\1&1&1}\Vec{\pi}=\myvec{0\\0\\0\\1}
\\
    \implies\xleftrightarrow{R_{1}=\cfrac{R_{1}}{-3}}\myvec{1&0&\cfrac{-2}{3}\\0&-3&4\\0&3&-4\\1&1&1}\Vec{\pi}=\myvec{0\\0\\0\\1}
\\
    \implies\xleftrightarrow{R_{2}=\cfrac{R_{2}}{-3}}\myvec{1&0&\cfrac{-2}{3}\\0&1&\cfrac{-4}{3}\\0&3&-4\\1&1&1}\Vec{\pi}=\myvec{0\\0\\0\\1}
\\
    \implies\xleftrightarrow{R_{3}=R_{3}-3R_{2}}\myvec{1&0&\cfrac{-2}{3}\\0&1&\cfrac{-4}{3}\\0&0&0\\1&1&1}\Vec{\pi}=\myvec{0\\0\\0\\1}
\\
    \implies\xleftrightarrow{R_{3}=R_{4} ,R_{4}=R_{3}}\myvec{1&0&\cfrac{-2}{3}\\0&1&\cfrac{-4}{3}\\1&1&1\\0&0&0}\Vec{\pi}=\myvec{0\\0\\1\\0}
\\
    \implies\xleftrightarrow{R_{3}=R_{3}-R_{1}}\myvec{1&0&\cfrac{-2}{3}\\0&1&\cfrac{-4}{3}\\0&1&\cfrac{5}{3}\\0&0&0}\Vec{\pi}=\myvec{0\\0\\1\\0}
\\
    \implies\xleftrightarrow{R_{3}=R_{3}-R_{2}}\myvec{1&0&\cfrac{-2}{3}\\0&1&\cfrac{-4}{3}\\0&0&1\\0&0&0}\Vec{\pi}=\myvec{0\\0\\\cfrac{1}{3}\\0}
\end{multline}
\begin{multline}
    \implies\xleftrightarrow{R_{2}=R_{2}+\frac{4}{3}R_{3}}\myvec{1&0&\cfrac{-2}{3}\\0&1&0\\0&0&1\\0&0&0}\Vec{\pi}=\myvec{0\\\cfrac{4}{9}\\\cfrac{1}{3}\\0}
\\
    \implies\xleftrightarrow{R_{1}=R_{1}+\frac{2}{3}R_{3}}\myvec{1&0&0\\0&1&0\\0&0&1\\0&0&0}\Vec{\pi}=\myvec{\cfrac{2}{9}\\\cfrac{4}{9}\\\cfrac{1}{3}\\0}
\end{multline}
\begin{align}
    \pi = \myvec{\cfrac{2}{9}&\cfrac{4}{9}&\cfrac{1}{3}}
\end{align}

\begin{align}
    \implies \lim_{n\to\infty}\myvec{P_{11}^{(n)}&P_{12}^{(n)}&P_{13}^{(n)}\\P_{21}^{(n)}&P_{22}^{(n)}&P_{23}^{(n)}\\P_{31}^{(n)}&P_{32}^{(n)}&P_{33}^{(n)}} = \myvec{\cfrac{2}{9}&\cfrac{4}{9}&\cfrac{1}{3}\\\cfrac{2}{9}&\cfrac{4}{9}&\cfrac{1}{3}\\\cfrac{2}{9}&\cfrac{4}{9}&\cfrac{1}{3} }
\end{align}
\begin{align}
    P_{11}^{(n)}=\cfrac{2}{9}
\\
    P_{13}^{(n)}=\cfrac{1}{3}
\end{align}
Hence , Options (i) and (iv) are correct.

\end{enumerate}




